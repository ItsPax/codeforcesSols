\documentclass{article}

\usepackage{amsmath}
\usepackage{amssymb}

\author{Alexander Sieusahai; 1495197}
\date{Due Sept.27/17}
\title{Assignment 1 Solutions}

\begin{document}

\maketitle

\textbf{Question 1;} For each of the three given functions decide whether it is a solution of the respective ODE: \\

\tab i)

\[ y_1(x) = e^x, y'_1(x) = e^x \rightarrow e^x \neq e^x+e^x \]
Therefore $y_1(x)$ is not a solution of the DE.\\

\[y_2(x)=xe^x, y'_2=x+xe^x \rightarrow e^x+xe^x = xe^x+e^x \]
Therefore $y_2(x)$ is a solution to the DE.\\

\[y_3(x)=x, y'_3(x)=1 \rightarrow 1 \neq x+e^x \]
Therefore $y_3(x)$ is not a solution to the DE.\\
\\
\tab ii) 
\[ y_1(x)=-\frac{x^2}{4}, y'_1(x) = -\frac{x}{2} \rightarrow  \frac{x^2}{4}+x(-\frac{x}{2})+\frac{1}{4}x^2 = \frac{x^2}{2} - \frac{x^2}{2} = 0 \]
Therefore $y_1(x)$ is a solution to the DE.\\

\[y_2(x)=x+1, y'_2(x) = 1 \rightarrow 1 + x(x+1) -(x+1) \neq 0 \]
Therefore $y_2(x)$ is not a solution to the DE.\\

\[y_3(x)=x^2+1, y'_3(x)=2x \rightarrow 4x^2+2x^2-(x^2+1) \neq 0 \]
Therefore $y_3(x)$ is not a solution to the DE.\\

\tab iii)

\[y_1(x)=x \rightarrow y''(x) DNE \]
Therefore, $y_1(x)$ is not a solution to the DE.

\[y_2(x)=x^2, y'_2(x) = 2x, y''_2(x) = 2 \rightarrow (1-x^2)(2) - (x^2)(2x) + 2 =2-2x^2-2x^2+2 = 4-4x^2 \neq 0 \]
Therefore, $y_2(x)$ is not a solution to the DE.

$
y_3(x) = (1-x^2)^{1/2}, y'_3(x) = x(1-x^2)^{-1/2}, y''_3(x) = -(1-x^2)^{-3/2} \rightarrow \\
(1-x^2)(-1)(1-x^2)^{-3/2} + x(x)(1-x^2)^{-1/2} + (1-x^2)^{1/2}  \\
= -(1-x^2)^{-1/2} + x^2{1-x^2)^{-1/2} + (1-x^2)^{1/2} = -(1-x^2)(1-x^2)^{-1/2}+(1-x^2)^{1/2} = 0 
$
Therefore, $y_3(x)$ is a solution to the DE.\\


\textbf{Question 2;} Solve the IVP for y=y(x),
\[ y'=6y-3y^2, y(1)=1, \]
and determine $lim_{x\to\infty} y(x)$. Re-do your calculation with $y(2) = 3$ instead of $y(1) = 1$. \\
\\
$
\frac{dy}{dx} = 6y-3y^2 \rightarrow \frac{dy}{6y-3y^2} = dx \\
\text{Using partial fraction decomposition:} \\
\frac{1}{6y-3y^2} = \frac{A(2-y)}{(3y)(2-y)} + \frac{B(3y)}{(3y)(2-y)} \\
\text{Let A = 1/2 and B = 1/6.} \\
\frac{1}{6y-3y^2} = \frac{1}{6y} - \frac{1}{6(2-y)} \\
\text{It then follows that:} \\
\int \frac{1}{6y-3y^2} = \int \frac{1}{6y} - \int \frac{1}{6(2-y)} \\
= \frac{ln|y|}{6} - \frac{ln|2-y|}{6}+C \\
\therefore \int \frac{dy}{6y-3y^2} = \frac{ln|y|}{6} - \frac{ln|2-y|}{6} + C = x \\
\\
\text{Rearranging for y=f(x) format below:} \\
x-c= (1/6) (ln|y|-ln|2-y|) \\
6(x+c)=ln|\frac{y}{2-y}| \\
e^{6(x+c)} = \frac{y}{2-y} \\
2e^{6(x+c)}-ye^{6(x+c)} = y \\
\frac{2e^{6x+6c}}{1+e^{6x+6c}} = y \\
\\ \text{Using the initial values} (x,y) = (1,1), \\
1 = \frac{2e^{6-c}}{1+e^{6-c}} \\
2e^{6-6c} = 1 + e^{6-6c} \\
e^{6-6c} = 1 \\
6-6c = 0 \\
c = 1 \\
\therefore \text{The function when } y(1) = 1 \text{ is,}
\\
y = \frac{2e^{6x-6}}{1+e^{6x-6}} \\
\\
\text{Using the initial conditions } (x,y) = (2,3)
\\
\frac{2e^{12-6c}}{1+e^{12-6c}}=3 \\
2e^{12-6c} = 3+3e^{12-6c} \\
-e^{12-6c} = 3 \\
\\
\text{The above equation has no real solution for c}.
$

\\
\textbf{Question 3;} Find the general solution of $y' - \frac{2xy}{x^2+1} = 1.$\\
\\
Begin by multiplying $\mu$ on both sides.
\[\mu y' - \frac{2xy}{x^2+1}\mu = \mu \]
Notice $(\mu y)' = y'\mu + \mu'y. Then:
\[\mu y' -\frac{2xy\mu}{x^2+1} = \mu y' + y \mu' \]
\[ \rightarrow -\frac{2xy\mu}{x^2+1} = y\mu' \]
\[ \rightarrow \int \frac{d\mu}{\mu} = -\int \frac{2x}{x^2+1} \]
Let $a = x^2 + 1; da = 2x$. Then
\[ ln(\mu) = -\int \frac{da}{a} \]
\[ \rightarrow ln(\mu) = ln(a^{-1}) \]
\[ \rightarrow \mu = \frac{1}{x^2+1} \]
It follows that:
\[ \int (\mu y)' = \int(x^2+1)^{-1} \]
Thus:
\[ y = (x^2+1)(\arctan(x)+c) \]


\\
\textbf{Question 4;} Determine the most general function $N = N(x,y)$ such that the equation
\[(ye^{xy}-4x^3y+2)dx + N(x,y)dy = 0 \]
is exact.\\
\\
Using \textbf{Theorem 2.16}:\\
Suppose N(x,y) is twice continuously differentiable in $\omega \subset \R^2$. Suppose the equation above is exact. It then follows that:
\[ \frac{\partial M}{\partial y} = \frac{\partial N}{\partial x} \]
\[ e^{xy}+xye^{xy}-4x^3 = \frac{\partial N}{\partial x} \]
Treating $y$ as a constant and taking the integral of $x$ on both sides leads to:
\[ \int e^{xy} +\int yxe^{xy} - \int 4x^3 = N(x,y) \]
The first integral and third integral are trivial to solve. The third integral requires technique. I will choose to use integration by parts.\\
Let $u = yx, du = y, dv=e^{yx}, v = \frac{e^{xy}}{y}. It follows that: 
\[\int yxe^{xy} = xe^{xy}-\int \frac{e^{xy}}{y}*y = xe^{xy}- \frac{e^{xy}}{y} \]
Combining all integrals to form N(x,y):
\[\frac{e^{xy}}{y}+xe^{xy}-\frac{e^{xy}}{y}-x^4 = N(x,y) \rightarrow xe^{xy}-x^4 = N(x,y) \]
Therefore, $N(x,y) =xe^{xy}-x^4$. \\


\\
\textbf{Question 5;} Solve the IVP $(2y-x)y'+2x = y$ using the initial conditions $y(1) = 3$. \\
\\
Rearranging the equation:
\[ (2y-x)y' + 2x = y \rightarrow (2y-x)dy + (2x-y)dx \]
Let $M(x,y)=2x-y$, and let $N(x,y)=2y-x$. It follows that:
\[\frac{\partial N}{\partial x} = -1 = \frac{\partial M}{\partial y} \]
The equation is exact. By \textbf{theorem 2.16}, there exists a function $F$ such that:
\[ \frac{\partial F}{\partial y} = N = 2y-x \rightarrow F = y^2-xy+g(x) \]
It follows that:
\[ \frac{\partial F}{\partial x} = M = 2x-y = \frac{\partial y^2-xy+g(x)}{\partial x} = -y+g'(x) \]
Then:
\[2x-y = -y + g'(x) \rightarrow 2x = g'(x) \rightarrow g(x) = x^2+k  \]
Substituting $g(x)$ into F:
\[F=y^2-xy+x^2+k\]
Therefore, the solution to the DE in implicit form is:
\[y^2-xy+x^2=c\]
Using the initial conditions $(x,y) = (1,3)$:
\[3^2-3(1)+1^2 = 7 = c \]
Therefore, the solution to the IVP is:
\[y^2-xy+x^2=7\]


\textbf{Question 6;} Demonstrate first that the ODE $e^x(x+1)+(ye^6-xe^x)y' = 0$ is \textit{not} exact. Then show that $\mu(x,y)=e^{-y}$ is an integrating factor, i.e., multiplication by $\mu$ makes the ODE exact. Use this to find the general solution. \\
\\
Rearranging the equation:
\[e^x(x+1)+(-ye^y+xe^x)y' = 0 \rightarrow e^x(x+1)dx+(ye^y-xe^x)dy = 0\]
Let $N=ye^y-xe^x$, and $M=e^x(x+1)$. It follows that:
\[ \frac{\partial M}{\partial y} = 0, \frac{\partial N}{\partial x} =xe^x+e^x \]
It is obvious that they are not equivalent, so the equation is not exact. However, if the equation is multiplied by $\mu(x,y)=e^{-y}$:
\[(e^{x-y}(x+1))dx+(y-xe^{x-y})dy=0 \rightarrow \frac{\partial N}{\partial x} = -xe^{x-y}-e^{x-y} = \frac{\partial M}{\partial y} \]
Since the equation is now exact, it follows that:
\[ \frac{\partial F}{\partial y} = y-xe^{x-y} \rightarrow F = y^2+xe^{x-y}+g(x)\]
Note that:
\[ \frac{\partial F}{\partial x} = xe^{x-y}+e^{x-y}+g'(x)=e^{x-y}(x+1)\]
It then follows that:
\[ g'(x) = 0 \rightarrow g(x) = c \]
It follows that:
\[F=\frac{y^2}{2}+xe^{x-y}+c\]
Then, the general solution is:
\[\frac{y^2}{2}+xe^{x-y}=c_1\]


\textbf{Question 7;} Find the general solution of $2xydx+(y^2-3x^2)dy=0$.\\
\\
Begin by multiplying the equation by the integrating factor $\mu$:
\[ \mu 2xydx+ \mu(y^2-3x^2)dy=0 \]
Assume that $\frac{\partial \mu (2xy)}{\partial x} = \frac{\partial \mu (-3x^2)}{\partial y}. Then:
\[\frac{\partial \mu}{\partial y}(2xy) + \mu(2x) = \frac{\partial \mu}{\partial x}(y^2-3x^2)+\mu(-6x)\]
\[\frac{\partial \mu}{\partial y}(2xy)-\frac{\partial \mu}{\partial x}(y^2-3x^2)=\mu(-8x)\]
Assume $\mu$ is a function of $y$ only.
\[ \frac{d\mu}{dy}=\mu(-8x) \rightarrow \frac{d\mu}{\mu} = \frac{-4dy}{y} \rightarrow ln(\mu) = -4ln(y) \rightarrow \mu = y^{-4} \]
Since the equation is now exact, it then follows that:
\[ \frac{\partial F}{\partial y} = y^{-2}-3x^2y^{-4} \rightarrow F = -y^{-1}+x^2y^{-3}+g(x) \]
It then follows that:
\[ \frac{\partial F}{\partial x} = 2xy^{-3} +g'(x) = 2xy^{-3} \rightarrow g(x) = c\]
Therefore, the general solution to the DE is:
\[-y^{-1}+x^2y^{-3}=c \]

\end{document}

