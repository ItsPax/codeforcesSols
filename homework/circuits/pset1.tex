\documentclass{article}

\usepackage{amsmath}
\usepackage{SIunits}
\usepackage{tikz}

\title{Problem Set 1 Solutions}
\date{Due on September 26th, 2017}
\author{Alexander Sieusahai; 1495197}

\begin{document}

\maketitle

\textbf{Question 1:} How much energy is given to an electron as it flows through a 6V battery from the positive to the negative terminal? \\
\\
This will be solved with unit analysis. Since 1 \volt is the potential difference between two points on a conducting wire, a test charge moving in the positive direction (as stated in question) will gain 1 \volt. It follows that:

\[ \frac{6\joule}{\coulomb} \cdot \frac{1\coulomb}{(6.242\cdot10^{18}e)} = 9.612\cdot10^{-19}\joule \]



\textbf{Question 2:} Suppose you have a 2000mAh (milliamp-hour) capacity rechargeable batter which is completely discharged. \\

\tab a) If the battery provides a constant voltage of 3.8\volt, what is the capacity of the battery in Watt-hours?\\
\\
Again, by unit analysis: 
\[ 2Ah \cdot \frac{3.8\joule}{\coulomb} \cdot \frac{\coulomb}{As} = \frac{7.6\joule h}{s} = 7.6Wh \]

\tab b) How long will it take to charge the battery to full chrage using a PC USB port which delivers 500 mA charge current? Assume 100\% efficiency in charging. \\
\\
Using unit analysis:
\[2Ah \cdot \frac{1}{0.5A} = 4h \]
Therefore it takes 4 hours to charge the battery.
\\


\tab c) How long will it take to charge the battery to full chrage using a wall charger capable 2 A charge current? Assume 100\% efficiency in charging. \\
\\
Using unit analysis:
\[2Ah \cdot \frac{1}{2A} = 1h \]
Therefore it takes 1 hour to charge the battery.
\\


\tab d) How much charge expressed in Coulombs does the battery contain when fully charged?\\
\\
Using unit analysis:
\[2Ah \cdot \frac{3600s}{h} \cdot \frac{\coulomb}{As} = 7200\coulomb \]


\textbf{Question 3:} The charge flowing into the box is shown in the graph in Figure 1. Sketch the current i(t) and power absorbed by the box on the interval [0,10] ms.\\
\\
Look to see graph attached to the back of this document.\\


\textbf{Question 4:} Find the power that is absorbed or supplied by the circuit elements in Figure 2.\\
\\
\tab a) \\
The independent current source supplies 40W of power.\\
Circuit element 1 absorbs 12W of power.\\
The independent voltage source absorbs 28W of power.\\
\\
\tab b) \\
The independent current source supplies 64W of power.\\
Circuit source 1 absorbs 32W of power.\\
The dependent voltage source absorbs 32W of power.\\


\textbf{Question 5:} Find \textit{V\textsubscript{x}} in the network in Figure 3 using Tellegen's theorem.\\
\\
Since all elements are in series, they share the same current. So, the sum of the voltages when considering if they are supplying or absorbing power follows.\\
Going from left to right:
\[6+12-24+V_x+16-18=0 \rightarrow V_x = 0 \]
\\
\textbf{Question 6:} Find $I_x,I_y,$ and $I_z$ in the network shown in Figure 4.
\\
Using the bottom center node and KCL:
\[ -2 + 4 + I_z = 0 \rightarrow I_z = -2mA \]
Using the top center node and KCL:
\[12-3-I_x = 0 \rightarrow I_x = 9mA \]
Using the left center node and KCL:
\[ -I_z - I_y - 12 = 0 \rightarrow 2-I_y-12=0 \rightarrow I_y=-10mA \]
\\
\textbf{Question 7:} Find $V_{ab}$ in the network.
\\
Using KVL:
\[ -15 +15i = 0 \rightarrow i = 1A \]
The calculation for $V_{ab}$ follows:
\[ -15+11i = V_{ab} \rightarrow -15+11=-4=V_{ab} \]
\\
\textbf{Question 8:} Find the power suppiled by each source in the circuit.
\\
Since current distributes proportionally to all resistors and all are in parallel, the circuit can be reduced to one resistor of resistance:
\[ R_{eq} = \frac{1}{1+1/2+1/5} \rightarrow \frac{10}{17}\Ohm = R_{eq} \]
Using KCL at any remaining node results in 2A of current flowing through the resistor. It follows that:
\[ V_{resistor} = iR = 2(\frac{10}{17}) = \frac{20}{17} \]
Due to KVL, the voltage at the simplified source is:
\[ V_s = -\frac{20}{17} \]
Since voltage is constant between the two current sources:
\[ P_{4mA} = (V_s)(0.004) = 4.704mW \]
\[ P_{2mA} = (-V_s)(0.002) =-2.352mW \]
Since the 4mA source would have to be powering the circuit, and the resistor is taking energy:
The 4mA source delivers 4.704mW to the circuit, and the 2mA source takes 2.352mW of power from the circuit. 

\end{document}
